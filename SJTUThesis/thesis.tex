%# -*- coding: utf-8-unix -*-
%%==================================================
%% thesis.tex
%%==================================================

% 双面打印
% \documentclass[doctor, fontset=adobe, openright, twoside]{sjtuthesis}
% \documentclass[bachelor, fontset=adobe, openany, oneside, submit]{sjtuthesis}
\documentclass[master, fontset=adobe, twoside, openright, submit]{sjtuthesis}
% \documentclass[%
%   bachelor|master|doctor,	% 必选项
%   fontset=adobe|windows,  	% 只测试了adobe
%   oneside|twoside,		% 单面打印,双面打印(奇偶页交换页边距,默认)
%   openany|openright, 		% 可以在奇数或者偶数页开新章|只在奇数页开新章(默认)
%   zihao=-4|5,, 		% 正文字号:小四、五号(默认)
%   review,	 		% 盲审论文,隐去作者姓名、学号、导师姓名、致谢、发表论文和参与的项目
%   submit			% 定稿提交的论文,插入签名扫描版的原创性声明、授权声明
% ]

% 逐个导入参考文献数据库
\addbibresource{bib/thesis.bib}
% \addbibresource{bib/chap2.bib}

\lstdefinestyle{myScalastyle}{
  frame=shadowbox,
  language=scala,
  commentstyle=\color{red!50!green!50!blue!50},%浅灰色的注释
  rulesepcolor=\color{red!20!green!20!blue!20},%代码块边框为淡青色
  keywordstyle=\color{blue!90}\bfseries, %代码关键字的颜色为蓝色,粗体
  showstringspaces=false,%不显示代码字符串中间的空格标记
  stringstyle=\ttfamily, % 代码字符串的特殊格式
  keepspaces=true, %
  breakindent=22pt, %
  numbers=left,%左侧显示行号
  stepnumber=1,%
  numberstyle=\tiny, %行号字体用小号
  basicstyle={\footnotesize\ttfamily}, %
  showspaces=false, %
  flexiblecolumns=true, %
  breaklines=true, %对过长的代码自动换行
  breakautoindent=true,%
  breakindent=4em, %
  aboveskip=1em, %代码块边框
  %% added by http://bbs.ctex.org/viewthread.php?tid=53451
  fontadjust,
  captionpos=t,
  framextopmargin=2pt,framexbottommargin=2pt,abovecaptionskip=-3pt,belowcaptionskip=3pt,
  xleftmargin=4em,xrightmargin=4em, % 设定listing左右的空白
  texcl=true,
  % 设定中文冲突,断行,列模式,数学环境输入,listing数字的样式
  extendedchars=false,columns=flexible,mathescape=true
  numbersep=-1em
}

\begin{document}

%% 无编号内容:中英文论文封面、授权页
%# -*- coding: utf-8-unix -*-
\title{分布式并行计算框架的shuffle优化}
\author{付周望}
\advisor{戚正伟教授}
% \coadvisor{某某教授}
\defenddate{2018年1月15日}
\school{上海交通大学}
\institute{软件学院}
\studentnumber{115037910032}
\major{软件工程}

\englishtitle{Efficient Shuffle Management with SCache for DAG
Computing Frameworks}
\englishauthor{\textsc{Zhouwang Fu}}
\englishadvisor{Prof. \textsc{Zhengwei Qi}}
% \englishcoadvisor{Prof. \textsc{Uom Uom}}
\englishschool{Shanghai Jiao Tong University}
\englishinstitute{\textsc{School of Software} \\
  \textsc{Shanghai Jiao Tong University} \\
  \textsc{Shanghai, P.R.China}}
\englishmajor{Software Engineering}
\englishdate{Jan, 2018}


\maketitle

\makeenglishtitle

\makeatletter
\ifsjtu@submit\relax
	\includepdf{pdf/original.pdf}
	\cleardoublepage
	\includepdf{pdf/authorization.pdf}
	\cleardoublepage
\else
\ifsjtu@review\relax
% exclude the original claim and authorization
\else
	\makeDeclareOriginal
	\makeDeclareAuthorization
\fi
\fi
\makeatother


\frontmatter 	% 使用罗马数字对前言编号

%% 摘要
\pagestyle{main}
%# -*- coding: utf-8-unix -*-
%%==================================================
%% abstract.tex for SJTU Master Thesis
%%==================================================

\begin{abstract}

大数据时代的到来使得分布式计算变得越来越普及。
为了快速地处理大规模的数据,有大量复杂的分布式并行计算框架被设计并使用,比如Hadoop MapReduce\cite{hadoop},Spark\cite{apachespark},Dryad\cite{dryad}, Tez\cite{tez}等。
这些分布式计算框架大多采用将用户计算逻辑用有向无环图(Directed Acyclic Graph, DAG)的方式呈现出来。
在执行DAG的每一个计算阶段时,这些计算框架大多采用了整体同步并行计算模型(Bulk-Synchronous Parallel, BSP)来对大数据进行分布式的并行批处理。

在这些相邻的计算阶段之间,shuffle,或者说跨网络的多对多分块数据的读写满足了计算逻辑对于不同数据的依赖。与此同时,shuffle的过程也带来了大量的网络数据传输。
受限于计算任务对于数据的依赖和本身低效率的设计实现,shuffle过程会给计算任务的性能带来巨大损失。
尤其是在一些需要大量shuffle数据的情境中,shuffle的开销甚至会成为整个应用的性能瓶颈。
更重要的是,这个问题在大多数分布式并行计算框架中都普遍存在。

为了提供一种具有普遍意义的shuffle优化方案,本研究抽取了这些系统在shuffle设计中存在的一些共性问题:1)粗粒度的硬件资源管理降低了资源的利用率和复用率。
2)同步滞后的shuffle读取既增加了计算任务执行时对shuffle网络传输的显式等待时间,又给网络带来一个瞬时的流量高峰。

针对以上问题,本文提出了S(huffle)Cache --- 一个开源的即用型系统来优化DAG计算过程中的shuffle阶段。
通过在计算阶段真正执行前提取表达计算逻辑的DAG以及其中的shuffle依赖关系,SCache可以将shuffle过程从DAG计算过程中独立出来,从而提供更细粒度的硬件资源管理。
与此同时,SCache通过提前异步的shuffle传输来解决目前同步滞后的shuffle读取过程。
此外,SCache还利用内存来实现对shuffle数据的缓存,进一步提升shuffle过程的效率。
为了实现以上的优化目标,本研究做出了以下主要贡献:

\begin{enumerate}
    \item 将shuffle过程从计算过程中解耦,使得shuffle过程独立到外部进行管理,从而实现了更细粒度的硬件资源管理。
    \item 结合应用的上下文对shuffle数据进行预取,既避免了同步数据读取给网络带来的压力,又能将大部分网络传输时间隐藏到计算阶段。
    \item 结合应用的上下文对shuffle数据进行内存缓存,进一步提升shuffle过程的效率。
    \item 根据现有的分布式计算框架shuffle的特点设计了相应的接口(API)。通用的接口设计使得优化能被应用到不同的分布式并行计算框架当中。
\end{enumerate}

基于以上阐述,本研究课题实现了SCache,同时修改了Apache Spark\cite{apachespark}对SCache进行适配。
并且通过仿真实验和Amazon AWS EC2集群上大规模数据测试来验证其优化效果。
在不同的数据集和测试程序的测试中,SCache能减少将近89\%的shuffle开销。
在TPC-DS的测试中,SCache的优化能给分布式SQL查询带来平均大约40\%的性能提升。

\keywords{\large 分布式DAG计算框架, Shuffle, 优化}
\end{abstract}

\begin{englishabstract}

Recent years have witnessed the widespread use of distributed computing in the big data area.
Numbers of sophisticated distributed data parallel computing frameworks, such as Hadoop MapReduce\cite{hadoop}, Spark\cite{apachespark}, Dryad\cite{dryad}, and Tez\cite{tez},
have been developed and deployed to accelerate the big data processing.
Most of these frameworks do the computing by transforming the application logic into Directed Acyclic Graph (DAG).
In order to increase the parallelism, each computing stage is usually managed according to the Bulk-Synchronous Parallel (BSP) model during the execution of DAG.

Shuffle, or the cross-network read and aggregation of partitioned data among tasks with data dependencies in the consecutive execution stages, 
usually brings in large network transfer. 
Due to the dependency constrains and the limited performance of disks and networks, execution of those descendant tasks could be delayed by inefficient shuffles. 
This delay can further slow down the whole application process. 
The performance degradation introduced by shuffle can become overwhelming in the shuffle intensive applications.
Moreover, the above deficiencies of shuffle generally exist in most of the DAG data parallel computing frameworks. 
In this paper, we extract the common issues in current shuffle mechanism: 
1) The coarse granularity resource management decreases the utilization and multiplexing of hardware resources.
2)The synchronized shuffle read increases the explicit network waiting time during task execution and brings a network burst which further slows down the shuffle read itself.

Based on the above observations, we present S(huffle)Cache --- an open source plug-in system that particularly focuses on shuffle optimization in frameworks defining jobs as DAGs. 
By extracting and analyzing the DAGs and shuffle dependencies prior to the actual task execution, 
SCache can take full advantage of the fine granularity resource management and system memory to accelerate the shuffle process. 
Meanwhile, SCache manages the shuffle data out of the frameworks and transfers data asynchronously, which helps overlap the network transfer time and avoid network burst.
In addition, SCache provides an application-context-aware in-memory shuffle data management scheme to further accelerate the shuffle process.  
In order to achieve the optimizations, we make following contributions:

\begin{enumerate}
    \item Decouple the shuffles and manage them out of the DAG data parallel computing frameworks so that the shuffle data management can become more efficient.
    \item Implement the shuffle data pre-fetch with application context so that the network burst can be avoided and the network transfer time can be overlapped in execution phases.
    \item Implement the application-context-aware in-memory shuffle data management to accelerate the shuffle process.
    \item Design and implement the general APIs for the DAG data parallel computing frameworks so that the optimizations can be applied easily.
\end{enumerate}

We have implemented SCache and customized Apache Spark\cite{apachespark} to use it as the external shuffle service and co-scheduler. 
The performance of SCache is evaluated with both simulations and testbed experiments on a 50-node Amazon EC2 cluster.
Those evaluations have demonstrated that, by incorporating SCache, the shuffle overhead of Spark can be reduced by nearly 89\%, 
and the overall completion time of TPC-DS queries improves 40\% on average.


\englishkeywords{\large Distributed DAG frameworks, Shuffle, Optimization}
\end{englishabstract}



%% 目录、插图目录、表格目录
\tableofcontents
\listoffigures
\addcontentsline{toc}{chapter}{\listfigurename} %将插图目录加入全文目录
\listoftables
\addcontentsline{toc}{chapter}{\listtablename}  %将表格目录加入全文目录
\listofalgorithms
\addcontentsline{toc}{chapter}{算法索引}        %将算法目录加入全文目录

% \include{tex/symbol} % 主要符号、缩略词对照表

\mainmatter	% 使用阿拉伯数字对正文编号

%% 正文内容
\pagestyle{main}
%# -*- coding: utf-8-unix -*-
%%==================================================
%% chapter01.tex for SJTU Master Thesis
%%==================================================

%\bibliographystyle{sjtu2}%[此处用于每章都生产参考文献]
\chapter{绪论}
\label{chap:intro}

大数据时代的到来和云计算的迅猛发展使得用分布式计算的方式来处理海量数据的方案变得水到渠成。
分布式并行计算框架的出现,特别是基于有向无环图(DAG)的方式来表达计算逻辑的分布式计算框架,进一步简化了大数据处理的过程。
也正是如此,使得分布式计算框架在短时间内获得了大量的普及。
虽然如何使得分布式计算变得更高效一直是近几年的研究热点,但是大量工作都集中在优化计算阶段的方向。
而对于其中shuffle阶段则关注较少。但是不能忽视的是,在许多场景下,shuffle这种I/O密集型的操作甚至会成为整个分布式计算应用的性能瓶颈。
本文针对现有的基于DAG的分布式计算框架的shuffle特点,提出了一种通用,高效的shuffle优化方案。
本文首先会介绍基于DAG分布式计算框架以及其优化的研究背景,然后阐述本文的优化目标以及国内外相关研究现状,最后简单介绍本文的结构组成。

\section{研究背景}

大数据时代的到来使得企业要处理的数据量远远超过了一台机器的处理性能。
在分布式计算普及之前,企业只能通过不断升级昂贵的超级计算机的性能来满足指数级增长的数据量。
然而随着Google公开了MapReduce\cite{mapreduce}的并行计算模式之后,分布式并行计算逐渐进入了蓬勃发展的阶段。
相对于昂贵而复杂的大型机,分布式计算能使用造价较低的商用机,并通过网络组合成集群,从而提供与超级计算机相匹配的运算能力来对大数据进行批处理。
最近几年,更是有大量的分布式计算框架在学术界和工业界得以发表和公开同时也有大量的计算框架被部署到企业的生产环境中,成为大数据生态系统中最重要的一个组件。
其中应用最广泛的就是Hadoop MapReduce\cite{mapreduce},Spark\cite{apachespark}和Tez\cite{tez}等。

\begin{figure}[!htp]
	\centering
	\includegraphics[width=\textwidth]{dagOverview.pdf}
	\bicaption[fig:dagOverview]{分布式DAG计算框架执行示意图}{分布式DAG计算框架执行示意图}{Fig}{Overview of DAG computing frameworks}
\end{figure}

如图\ref{fig:dagOverview}所示,这些计算框架通过有向无环图(DAG)的方式来表达用户应用逻辑。
在执行DAG的过程中,对于每一个计算阶段(stage)采用了整体同步并行计算模型(BSP)来对大数据进行分布式的并行批处理。
在一个计算阶段,框架会将数据分块,并且对于每一块数据应用相同的用户定义计算来进行并行处理,既图\ref{fig:dagOverview}中的task。
在DAG的执行过程中,每一个任务都会有对上一个任务的数据依赖。这个数据依赖可以被分成两个类型:完全依赖和部分依赖。
完全依赖指的是当前计算阶段的一个任务所需要的输入数据完全依赖于上一个计算阶段的一个或几个任务的所有输出数据,比如图\ref{fig:dagOverview}中黑色箭头所表示的依赖关系。
部分依赖,也就是本文中提到的shuffle,则表示当前计算阶段的一个任务所需要的输入数据依赖于上一个计算阶段的多个任务的部分数据,比如图\ref{fig:dagOverview}中绿色的部分。
通常而言,对于产生shuffle数据的一端,我们称之为映射阶段(Map Stage),而对于接收shuffle数据的一端,我们称之为规约阶段(Reduce Stage)。
对于完全依赖,目前已经有了好的解决方案。比如Spark\ref{apachespark}中RDD\ref{spark}的解决方案,就通过将连续的完全依赖的计算阶段合并成一个大的计算阶段,并且通过内存计算的加速来优化了这部分数据依赖。

但是对于shuffle的依赖,虽然这些DAG计算框架在设计上存在很多不同,但是他们都是通过多对多的网络数据传输方式来实现。
而shuffle阶段由于受限于I/O设备性能的限制(磁盘,网络等),会对整个端到端的应用执行性能带来很大的额外开销。
虽然近几年针对分布式DAG计算框架的计算阶段学术界和企业界都提出了很多优化方案\cite{pacman, babu, quincy, sync},但是对于shuffle阶段在实际应用中的优化却一直很不理想。
比如在Facebook公开的一个MapReduce运行数据分析中,shuffle平均占到了所有任务完成时间的33\%。
对于需要大量shuffle的一些任务,shuffle的开销最多可以占到整个任务完成时间的70\%\cite{managing}。

\section{研究内容}

本文通过对这些计算框架的研究,发现制约shuffle性能的主要是由于缺乏对于不同类型的硬件资源的细粒度的管理和调度。
在目前的DAG计算框架调度算法中,对于一个计算阶段的每一个任务,DAG计算框架的调度器都会分配集群中一部分固定的硬件资源,包括CPU,内存,磁盘和网络等。
为了简化调度算法,这些资源被捆绑成一个slot来进行粗粒度的管理和分配。
这种粗粒度的调度算法虽然简化了调度过程,但是也引入无法充分利用硬件资源的问题。
比如,当一个任务进行CPU内存密集的数据计算时,该slot所占用的I/O资源就会被限制。反之,当任务进行I/O密集的shuffle过程时,CPU和内存等计算资源就会被闲置。

除此之外,在shuffle阶段不可避免得会引入多对多的网络数据传输。
在目前的分布式DAG计算框架中,此阶段的网络数据传输也没有得到很好的管理。
当reduce阶段被调度并启动之后,集群中会有多个reduce任务同时启动,并且几乎同时通过网络来从远程节点获取数据。
这种几乎同步的数据传输模式会给集群的网络带来一个瞬时的流量高峰。
而当带宽有限的情况下,这种瞬时的高峰极易造成网络的拥塞,从而进一步减慢了数据传输的速率。

更糟糕的是,上述发现的问题存在于大部分主流的分布式DAG计算框架当中。所以仅仅只是针对其中某一个框架提出解决方案并不能很好的缓解shuffle给分布式计算带来的性能开销。

为了给shuffle过程提供一个具有普适性的优化方案,本文提出了S(huffle)Cache --- 一个开源的即插型shuffle管理系统来给不同的DAG计算框架提供高效的shuffle管理和优化。
具体来说,SCache通过提供跨框架的API设计,来接管在DAG计算过程中的shuffle阶段。
同时SCache采用了以下几点关键创新,来实现对于shuffle的高效管理和优化:

\begin{enumerate}
	\item 将shuffle从计算过程中的解耦。使得shuffle的过程独立到DAG计算框架外部进行管理,从而实现了更细粒度的硬件资源管理,提高硬件资源的复用率和利用率,进而加速shuffle过程。
	\item 结合应用的上下文对reduce任务进行预调度。采取了启发式算法,根据map阶段执行过程中的中间状态,结合应用的上下文逻辑和数据本地性等特征提前调度reduce阶段的任务。
	\item 对shuffle数据进行预取。在map执行阶段,根据启发式预调度算法的结果,对shuffle数据进行预取,既避免了同步数据读取给网络带来的压力,又能将大部分网络传输时间隐藏到计算的阶段。
	\item 采用了结合上下文的内存管理机制。根据DAG计算框架的任务调度策略,对不同的shuffle数据块设置优先级,同时提前将shuffle数据缓存在内存当中,加速shuffle数据的读取过程,提升任务计算性能。
	\item 设计了具有普适性的API。SCache不仅根据shuffle的读写为DAG计算框架设计了相应的API,同时也为DAG框架的调度器设计了相应提交shuffle相关元数据和获取预调度结果的API。
\end{enumerate}

\section{国内外研究现状}

目前国内外的对shuffle的优化工作主要分成三个方向:提前调度,延迟调度和纯网络层面的优化。

\textbf{提前调度}:Slow-start作为Hadoop MapReduce\cite{hadoop}中最经典的shuffle优化是提前调度的代表性方案。
Starfish\cite{starfish}通过对数据的采样来自动调整MapReduce中的系统参数,比如slow-start的比例,map和reduce任务的比例等待。
DynMR\cite{dynmr}通过动态的在map阶段末端启动reduce任务来减少对与shuffle数据的等待。
以上所有这些方案都没有将I/O操作从计算密集型的任务中解耦,因而仍然在slot中留下显示的I/O等待时间。
而且由于计算资源有限,提前启动reduce任务会占用有限的slot,减慢map阶段的执行。
所以在设定何时启动提前调度的参数时,会受到应用,输入数据,当前硬件资源等条件的影响,不仅参数设置困难,优化效果也会有较大波动。
iShuffle\cite{ishuffle}采用了讲shuffle从reduce阶段解耦的方式,并且提供了一个中心控制器来调度shuffle。但是这个方案并不能和好的处理对多个shuffle的依赖。
iHadoop\cite{ihadoop}采用了激进地提前调度多个接下来的计算阶段的任务,从而是的shuffle数据的预取成为可能。
Drizzle\cite{drizzle}也采用了提前调度任务的方式来实现shuffle数据的预取。
但是在我们研究过程中发现,随机激进地调度任务可能会破坏DAG计算框架的负载均衡,从而减慢应用的性能。

\textbf{延迟调度}:Delay Scheduling\cite{delay}采用延迟分配任务来获取更好的数据本地性,进而减少了shuffle阶段的网络数据传输。
ShuffleWatcher\cite{shufflewatcher}会在网络将近饱和的时候延迟shuffle数据的获取。同时它能在调度任务时获取更好的本地行。
Quincy\cite{quincy}和Fair Scheduling\cite{preemptive}都可以通过优化map任务的调度来获取更好的shuffle数据本地性。
但是以上这些工作都不能消除在计算任务中显示的I/O操作。更重要的是,他们的优化效果会因为网络的性能好坏和计算数据的不同分布而产生波动。

\textbf{网络层优化}:Varys\cite{varys}和Aalo\cite{aalo}都能结合应用层信息提供shuffle在网络传输时的优化,从而加快shuffle的传输过程。
虽然这些优化仅仅局限在shuffle过程中的网络传输过程,但是这些共工作可以为本研究的优化提供进一步的性能提升。

\section{文章结构}

本文余下内容结构如下:

第二章介绍shuffle在分布式并行DAG计算框架中的特点以及相关背景。通过对shuffle特性的分析,来挖掘其可优化的空间和制定具体优化方案。

第三章详细介绍了SCache的具体设计与实现,包括SCache的架构,shuffle调度算法设计,接口设计以及内存管理策略。
同时还会介绍SCache和Spark协同工作的实例。

第四章对SCache在Spark平台上的优化效果进行了实验和分析。
通过仿真实验和Amazon AWS EC2上虚拟机集群的测试,验证了SCache在shuffle上的优化效果。

第五章对全文进行了总结,并且对未来的工作方向做出展望。








%# -*- coding: utf-8-unix -*-
%%==================================================
%% chapter02.tex for SJTU Master Thesis
%%==================================================

\chapter{Shuffle特性分析}
\label{chap:observations}

本章首先分析了shuffle在目前分布式DAG计算框架中的特性,然后通过对其特征的研究,结合分布式DAG计算框架的特性,探索其优化的方向。

\section{Shuffle的特点}
在分布式DAG计算框架中,shuffle被用来实现一个在计算过程中任务的多对多的部分数据依赖。
同时也代表了一次计算集群内部节点多对多的网络数据传输。
为了便于理解,本文用map阶段来代表产生shuffle数据的DAG计算阶段,用reduce阶段代表获取shuffle数据作为输入的计算阶段。

在图\ref{fig:workflow}中,我们呈现了一个shuffle在现有分布式DAG框架中局部的工作流程, 该流程由两个map任务和一个reduce任务组成。
如图\ref{fig:workflow}所示,其中的shuffle write表示一个map任务将计算产生的中间结果写入到本地磁盘的过程。
这个过程会在map任务完成对所有输入数据的计算之后开始。
在每一个map任务内部,计算框架会根据用户定义的计算方法和现阶段的数据,对于其中一个数据分区(partition)进行计算。
计算完成之后,map任务就会根据用户定义的分区函数(比如哈希分区或者排序分区),将计算产生的中间结果(键值对)进行分块存储。
其中存储这些代表中间结果的键值对的数据块的数目等同于下一个计算阶段(reduce)的任务数目。
这些分块的数据即在shuffle阶段通过网络传输的数据。这些数据块往往会在分块结束之后被写入本地磁盘。

在reduce阶段开始真正的计算之前,首先要通过网络从远程的节点的磁盘上拉取map阶段产生的属于该任务输入的部分shuffle数据块,即图\ref{fig:workflow}中的shuffle read部分。

可以看到在传统的分布式DAG计算框架中,受限于粗粒度的资源调度机制,无论是shuffle write还是shuffle read,这些I/O密集型操作都是由计算任务(task)来执行的。
而计算任务在分布式DAG计算框架的调度中,是按照集群中计算资源的负载进行均衡调度的。
也就是说在分配计算任务时,分布式DAG计算框架的调度器默认其为计算密集型任务,因此会根据各个节点计算资源(比如CPU的空闲核数)来申请资源并调度任务。
但是任务被调度之后,却没有充分利用计算资源,反而在占用计算资源的情况下进行大量的I/O操作。
正因为这种粗粒度的调度和集成在计算任务中的I/O操作,使得硬件资源的利用率和复用性严重受损,进而损害了计算框架本身的性能。
有相关研究表明,Yahoo!公司中60\%的MapReduce工作和Facebook公司中20\%的MapReduce工作中存在大量的shuffle数据\cite{shufflewatcher}。
对于这些存在大量shuffle的工作中,shuffle传输所带来的延迟甚至会成为整个工作完成时间的瓶颈。
比如在一份Facebook公司公开的MapReduce任务日志中,shuffle过程的平均占用了整个工作完成时间的33\%。
对于那些存在大量shuffle的工作,shuffle的开销甚至最多占用了整个工作完成时间的70\%。

\begin{figure}[!htp]
	\centering
	\includegraphics[width=\textwidth]{../../PPoPP-2018/fig/workflow.pdf}
	\bicaption[fig:workflow]{传统分布式DAG计算框架工作流程与在SCache协同下工作流程的比较}{传统分布式DAG计算框架工作流程与在SCache协同下工作流程的比较}{Fig}{Workflow Comparison between Legacy DAG Computing
	Frameworks and Frameworks with SCache}
\end{figure}

\section{对shuffle的观察}

为了满足计算逻辑中的部分依赖,shuffle这个过程在分布式DAG的计算过程中是无法避免的。
但是我们能否通过改进shuffle的过程来减少其对计算框架性能的影响呢?
为了探索对其优化的可能性,我们在一个由5个Amazon AWS EC2 m4.xlarge\cite{aws}节点组成的集群中运行了一些具有代表性的Spark应用。
在运行这些应用的同时,我们测量了每个节点的CPU利用率,磁盘和网络的吞吐率,并且收集了每个应用的执行信息。
在图\ref{fig:util}中,我们展示了在运行一个Spark GroupByTest应用时一个节点上的硬件资源利用率随时间变化的结果。
Spark GroupByTest是一个包含两轮计算阶段,中间通过一次shuffle连接的简单应用。
图中的“execution”部分表示了从第一个计算任务启动到最后一个计算任务结束的时间,
“shuffle write”阶段表示从第一个任务shuffle数据块被写入开始到最后一个任务的最后一个数据块写入结束的时间,
“shuffle read and execution”则表示从第一个reduce任务开始获取shuffle数据并执行到所有reduce任务执行结束的时间。

从单节点的硬件资源率用率和任务执行的时间分布,结合剩余节点采集的数据,我们发现以下一些问题。

\begin{figure}[!htp]
	\centering
	\includegraphics[width=\textwidth]{../../PPoPP-2018/fig/util.pdf}
	\bicaption[fig:util]{运行包含一个shuffle的Spark应用时的硬件资源利用率}{运行包含一个shuffle的Spark应用时的硬件资源利用率}{Fig}{CPU utilization and I/O throughput of a node
	during a Spark single shuffle application}
\end{figure}

\subsection{粗粒度的资源分配}

在现有的硬件资源分配机制下,当一个slot被计算框架分配给一个任务的时候,只有在该任务执行结束,这个slot以及其对应的硬件资源才会被释放。
对于一个map任务的结束是在他完成了shuffle write之后。
而在reduce任务一端,当一个reduce任务占用一个slot之后,它首先需要的只是通过一些I/O操作来获取远程的shuffle数据。
而在图\ref{fig:util}中,可以看到shuffle数据的写入和shuffle数据的读取都是I/O密集型操作,在这两个阶段,slot中的CPU利用率很低。
反过来,在进行计算的过程中,节点上的I/O资源则被大量闲置。
由此可知,在目前以Spark为代表的分布式DAG计算框架分配的任务的不同阶段对硬件资源的需求是不一致的。
但是当前这种主流的粗粒度资源分配方式为了满足任务在不同阶段的需求,只能舍弃一部分硬件资源的率用率。
因此要改变这种资源与需求的不匹配,必须提供一种更细粒度的资源分配机制。

\subsection{同步滞后的shuffle读取}

在目前的主流分布式DAG计算框架中,大多采用了Bulk Synchronous Parallel(BSP)模型来控制每个计算阶段。
因此当图\ref{fig:util}中的reduce阶段被调度时,集群中所有节点都会几乎同时启动reduce任务。
而reduce任务启动之后首先需要通过网络来完成shuffle数据的读取。
结合图\ref{fig:util}可以发现,在“shuffle read”阶段,该节点的网络产生了一个瞬时的流量高峰,而此时集群中其他节点也会产生相应的流量高峰。
这种几乎同时的流量高峰的出现,会对整个集群的网络带来巨大的压力,极易造成链路上的拥塞。
而当网络检测到拥塞时,无论是传统的TCP\cite{tcp}还是更先进的DCTCP\cite{dctcp}都会降低传输速率。
同时拥塞的发生还可能导致网络中交换机的丢包和TCP的重传等。
这些因素都会减慢shuffle读取时网络传输的速率,延长了shuffle读取过程,进而损害了reduce阶段的性能。

\subsection{低效率的磁盘操作}
\label{subsec:size}

在整个shuffle过程中,至少存在两次磁盘操作,即map阶段的shuffle数据写入磁盘和reduce阶段从远程磁盘读取shuffle数据。
在前文中提到的粗粒度的资源分配机制使得计算任务中的I/O操作会阻塞CPU内存等计算资源的释放和利用。
而其中低效率的磁盘操作更加剧了这个阻塞带来的延迟。

为了解决磁盘读写阻塞带来的延迟,首先需要回答两个问题:shuffle数据在磁盘的持久化是否必要;如果持久化是必要的,那么能否通过异步的方式来隐藏磁盘操作的开销。
对于将shuffle数据写入磁盘的必要性,在当下数据中心内存日益增大的背景下,为了节约内存而采用的将shuffle数据写入磁盘的方式(spill)并没有那么强的必要性。
而且相较于应用的输入数据而言,shuffle数据作为中间结果,其体积要相对小很多。
比如Spark Terasort\cite{spark-tera}中,shuffle数据只占到了输入数据的25\%不到。
在一些研究中\cite{makingsense}的数据也表明,shuffle数据大约只占到了输入数据的10\% $\sim$ 20\%。
从另一个角度来看,目前出现了越来越多的基于内存的分布式存储系统,比如memcached\cite{memcached},Tachyon\cite{tachyon}(现在的Alluxio\cite{alluxio})以及RAMCloud\cite{ramcloud}等。
这个趋势间接表面了在目前的数据中心环境下,内存资源是足够用来存放这些数据的。

退一步说,即使受限于内存体积,使得对shuffle数据的磁盘写操作不可避免,但是由于DAG的存在,结合调度算法可以很容易就获取任务的调度顺序。
在这个前提下,即使内存资源十分紧张,shuffle的数据也可以通过异步预取的方式,提前从磁盘缓存在内存中。
因此我们认为,目前在shuffle过程中低效率的磁盘操作是可以被优化的。

\subsection{多轮任务执行}

在运行分布式DAG计算应用的时候,对于每一阶段的计算任务,无论是经验还是这些框架的手册都推荐在开发应用时将每个计算阶段并行的任务划分成多轮。
即对于一个有$s$个计算slot的计算集群而言,每个计算阶段的并行任务$t$:
\begin{equation}
	\label{eq:multi}
	t = k \times x \quad (k = 1, 2, 3, ...)
\end{equation}
比如Hadoop MapReduce的手册\cite{hadooptutorial}建议每个节点运行10-100个map任务(即等式\ref{eq:multi}中 $k \in [10, 100]$)。
同时该手册还建议运行时将任务的并发度调整到“$0.95$ 或 $1.75 \times$ 节点数 $\times$ 每个节点的最大容器数”
。
Spark的配置手册\cite{sparkconf}也建议为节点上的每个CPU分配2-3个任务(即等式\ref{eq:multi}中 $k = 2, 3$)。

对于shuffle数据而言,当每一个map任务运行结束时,该任务所产生的shuffle数据就能被完整的获取。
于此同时,结合图\ref{fig:util},可以观察到在任务执行过程中,网络一直处于闲置状态。
因此在多轮任务执行的应用环境当中,如果shuffle传输的目的节点已知,那么shuffle数据便可以在map阶段的执行过程中进行预取。
这种多轮任务的属性则可以很好的将shuffle数据传输过程进行重叠,进而消除目前存在与reduce任务执行过程中的显示shuffle网络传输等待时间。

\section{本章小结}
通过以上的观察和分析,我们认为shuffle存在继续优化的空间。
具体来讲,就是通过解耦合的方式移除阻塞式的I/O操作,然后利用多轮任务执行的特性对shuffle数据进行预取,同时通过内存数据缓存和网络传输过程中的优化来进一步加速shuffle的读写。
基于上述优化目标,我们提出了SCache --- 针对分布式DAG计算框架shuffle过程的优化系统。
在图\ref{fig:workflow}的下半部分展现了DAG计算框架与SCache协同工作时的工作流程。
在shuffle write阶段,SCache通过内存拷贝的方式讲数据直接从任务的内存空间拷贝到预留的shuffle数据存储内存区域。
于此同时,磁盘的写操作将被省略。而由分布式DAG计算框架分配的slot资源也会在内存拷贝结束之后被立即释放。
这些缓存在内存中的shuffle数据会在SCache完成对reduce任务的预调度之后立刻进行网络传输。
这种数据预取方式既能将shuffle read阶段的时间很好的隐藏在map任务执行的过程中,又能避免传统BSP模型下,分布式DAG计算框架统一调度和执行reduce任务带来的同步瞬时大量的网络吞吐。

为了实现以上优化:
\begin{itemize}
	\item Shuffle的过程必须从计算任务中解耦,从而实现更细粒度的资源分配和更高的资源利用率。
	\item Reduce任务必须在map阶段就进行预调度,从而实现shuffle数据预取。同时在预调度的过程中不能真正启动reduce任务,以防其占用计算资源。
	\item 需要结合DAG计算逻辑实现精细化shuffle数据内存缓存管理,进一步加快shuffle读写速度。
	\item 为了实现跨框架的优化方案,shuffle过程必须被独立到计算框架外部进行管理,并且实现一个具有通用性的API。
\end{itemize}

\section{Shuffle Optimization}\label{opt}
This section presents the detailed methodologies to achieve three design goals. The out-of-framework shuffle data management is used to decouple shuffle from execution and provide a cross-framework optimization. Two heuristic algorithms (Algorithm \ref{hminheap}, \ref{mhminheap}) are used to achieve shuffle data pre-fetching without launching tasks.

\subsection{Decouple Shuffle from Execution}
During map tasks, the partitioner takes a set of key-value pairs as input and calculates the partition number for each of them by applying pre-defined the partition function. After that, it stores all key-value pairs in the corresponding data blocks. Each of the block contains the key-value pairs for one reduce partition. At the same time, blocks will be spilled to disk. In order to avoid blocking the slot by disk operation, we use memory copy to hijack shuffle data from map tasks. By doing this, a slot can be released as soon as a task finishes CPU intensive computing. 
From the perspective of reduce task, shuffle read is decoupled by pre-fetching shuffle data to local SCache client before reduce tasks start.

\begin{figure}
	\centering
	\includegraphics[width=0.75\linewidth]{fig/sim}
	\caption{Stage Completion Time Improvement of OpenCloud Trace}
	\label{fig:sim}
\end{figure}

\begin{figure}
	\centering
	\includegraphics[width=0.6\linewidth]{fig/reduce_cdf}
	\caption{Shuffle Time Fraction CDF of OpenCloud Trace}
	\label{fig:cdf}
\end{figure}
%To this end, all I/O operations are managed outside of the DAG framework, and the slot is occupied only by the CPU intensive phases of task.
\subsection{Pre-schedule with Application Context}
The pre-scheduling and pre-fetching start when the collected shuffle data exceeds the threshold.
This is the most critic step toward the optimization. The task --- node mapping is not determined until tasks are scheduled by the scheduler of DAG framework. Once the tasks are scheduled, the slots will be occupied to launch them. On the other hand, the shuffle data cannot be pre-fetched without the readiness of task --- node mapping.
To get rid of this dilemma, we propose a co-scheduling scheme with two heuristic algorithms (Algorithm \ref{hminheap}, \ref{mhminheap}). That is, the task --- node mapping is established ahead of DAG framework scheduler, and it is enforced to DAG scheduler before the real scheduling.

\begin{figure}
	\centering
	\includegraphics[width=0.9\linewidth]{fig/shuffle}
	\caption{Shuffle Data Prediction}
	\label{fig:shuffle}
	\vspace{-1em}
\end{figure}
% We explore several pre-scheduling schemes in different scenarios, and evaluate the performance calculating the improvement of reduce tasks completion time with trace of OpenCloud \cite{opencloudtrace}. We first emulate the scheduling algorithm of Spark to schedule the reduce tasks of one job, and take the bottleneck of the task set as the completion time. Then we remove the shuffle read time as the assumption of shuffle data pre-fetch and emulate under different schemes. The result is shown in \ref{fig:sim}.
% \begin{figure*}
% 	\centering
% 	\begin{minipage}{0.34\linewidth}
% 		\begin{figure}[H]
% 			\includegraphics[width=\textwidth]{fig/shuffle_size}
% 			\caption{Shuffle Size Comparing with Input Size}
% 			\label{fig:shuffle_size}
% 		\end{figure}
% 	\end{minipage}
% 	\begin{minipage}{0.65\linewidth}
% 		\begin{figure}[H]
% 			\begin{subfigure}{0.5\textwidth}
% 				\includegraphics[width=\linewidth]{fig/reduce_cdf}
% 				\caption{Shuffle Time Fraction CDF}
% 				\label{fig:cdf}
% 			\end{subfigure}	
% 			\begin{subfigure}{0.5\textwidth}
% 				\includegraphics[width=\linewidth]{fig/sim}
% 				\caption{Stage Completion Time Improvement}
% 				\label{fig:sim}
% 			\end{subfigure}	
% 			\caption{Emulate Result of OpenCloud Trace}
% 		\end{figure}
% 	\end{minipage}
% \end{figure*}
% \begin{figure}
\subsubsection{Problem of Random Mapping}\label{randomassign}
The simplest way of pre-scheduling is mapping tasks to different nodes randomly. It only guarantees that each node run same number of tasks. 
As shown in Figure \ref{fig:sim}, we use traces from OpenCloud \footnote{\label{fn:trace}http://ftp.pdl.cmu.edu/pub/datasets/hla/dataset.html} for the simulation to evaluate the impact of different pre-scheduling algorithms. The baseline (red dot line in Figure \ref{fig:sim}) is the stage completion time with Spark default scheduling algorithm. And then we remove the shuffle read time of each task, and run the simulation under three different schemes: random mapping, Spark FIFO, and our heuristic MinHeap.
Note that most of the traces from OpenCloud is shuffle-light workload as shown in Figure \ref{fig:cdf}. The average shuffle read time is 3.2\% of total reduce completion time.

Random mapping works well when there is only one round of tasks. But the performance collapses as the round number grows. It is because that data skew commonly exists in data-parallel computing \cite{skewtune, reining, gufler2012load}. Several heavy tasks might be assigned on the same node. This collision then slows down the whole stage and makes the performance even worse than the baseline. In addition, randomly assigned tasks also ignore the data locality between shuffle map output and shuffle reduce input, which might introduces extra network traffic in cluster.

\begin{figure*}
	\centering
	\begin{subfigure}[b]{0.32\linewidth}
		\includegraphics[width=\linewidth]{fig/hash_pre}
		\caption{Linear Regression Prediction of Hash Partitioner}
		\label{fig:hash_pre}
	\end{subfigure}
	\begin{subfigure}[b]{0.32\linewidth}
		\includegraphics[width=\linewidth]{fig/range_pre_sample}
		\caption{Linear Regression and Sampling Prediction of Range Partitioner}
		\label{fig:range_pre_sample}
	\end{subfigure}
	\begin{subfigure}[b]{0.32\linewidth}
		\includegraphics[width=\linewidth]{fig/prediction_relative_error}
		\caption{Prediction Relative Error of Range Partitioner}
		\label{fig:prediction_relative_error}
	\end{subfigure}
	\caption{Reduction Distribution Prediction}
	\label{fig:dis}
\end{figure*}

\subsubsection{Shuffle Output Prediction}\label{shuffleprediction}
The problem of random mapping was obviously caused by application context (e.g., shuffle data size) unawareness. Note that the optimal schedule decision can be made under the awareness of shuffle dependencies number, partition number, and shuffle size for each partition. The first two of them can be easily extracted from DAG information. The scheduling can be made with the "prediction" of shuffle size.

According to the DAG computing process, the shuffle size of each reduce task is decided by input data, map task computation, and hash partitioner. Each map task produces a data block for each reduce task. The size of each reduce partition can be calculated $reduceSize_i = \sum_{j=0}^{m} {BlockSize_{ji}}$ ($m$ is the number of map tasks). $BlockSize_{ji}$ represents the size of block which is produced by map $task_j$ for reduce $task_i$ (e.g., block `1-1' in Figure \ref{fig:shuffle}).

For the simple DAG applications such as Hadoop MapReduce \cite{mapreduce}, the $BlockSize_{ji}$ can be predicted with decent accuracy by liner regression model based on observation that the ratio of map output size and input size are invariant given the same job configuration \cite{ishuffle, predict}.

But the sophisticated DAG computing frameworks like Spark introduce more uncertainties. For instance, the customized partitioner might bring large inconsistency between observed map output blocks distribution and the final shuffle data distribution. In Figure \ref{fig:dis}, we use different datasets with different partitioners, and normalize the distribution to $0-1$ to fit in one figure. In Figure \ref{fig:hash_pre}, we use a random input dataset with the hash partitioner. In Figure \ref{fig:range_pre_sample}, we use a skew dataset with the range partitioner of Spark \cite{apachespark}.
The observed map outputs are randomly picked. As we can see, in hash partitioner, the distribution of observed map output is close to the final reduce input distribution. The prediction results also turn out to be good. However, this inconsistency results in a deviation in linear regression model.
% Several map outputs (marked as Map Output in Figure \ref{fig:shuffle}) are picked as observation objects to train the model and than predict the final reduce distribution.
To handle this inconsistency, we introduce another methodology named weighted reservoir sampling. The $BlockSize_{ji}$ can be calculated by
\begin{equation}
\label{equationsample}
\begin{aligned}
	BlockSize_{ji} &= {{InputSize_j \times \frac{sample_i}{s \times p}}} \\
	sample_i &= \text{number of samples for $reduce_i$}
\end{aligned}
\end{equation}
% The classic reservoir sampling is designed for randomly choosing \textit{k} samples from \textit{n} items, where \textit{n} is either a very large or an unknown number \cite{reservoir}. 
For each partition of map task, we use classic reservoir sampling to randomly pick $s \times p$ of samples, where $p$ is the number of reduce tasks and $s$ is a tunable number. After that, the map function is called locally to process the sampled data (\textit{sampling} in Figure \ref{fig:shuffle}). The final sampling outputs are collected with the $InputSize$ of each map partition which is used as the weight for each set of samples.

In Figure \ref{fig:range_pre_sample}, we set $s$ equals $3$, the result of sampling prediction is much better even in a very skew scenario. The variance of the normalization between sampling prediction and reduce distribution is because the standard deviation of the prediction result is relatively small compared to the average prediction size, which is $0.0015$ in this example. Figure \ref{fig:prediction_relative_error} further proves that the sampling prediction can provide precise result even in the dimension of absolute shuffle partition size. On the opposite, the result of linear regression comes out with large relative error.

To avoid extra overhead, the sampling prediction will be triggered only when the range partitioner or customized non-hash partitioner occurs. We will show the detail evaluation of sampling in the Section \ref{evaluation}.

During the phase of shuffle output prediction, the composition of each reduce partition is calculated as well. We define $prob_i$ as
\begin{equation}
\label{equationprob}
\begin{aligned}
	prob_i &= \max_{0 \leq j \leq m} \frac{BlockSize_{ji}}{reduceSize_i} \\
    m &= \text{number of map tasks}
\end{aligned}
\end{equation}
This parameter is used to achieve a better locality while performing shuffle scheduling.

\subsubsection{Heuristic MinHeap Scheduling}\label{h-minheap}
In order to balance load on each node while reducing the network traffic, we present a heuristic MinHeap scheduling algorithm (Algorithm \ref{hminheap}) for single shuffle.  
Heuristic MinHeap can be divided into two rounds. In the first round (i.e., the first $while$ in $SCHEDULE$), the reduce tasks are first sorted by size in a descending order. For hosts, we use a min-heap according to size of assigned tasks to maintain the priority. So that the tasks can be distributed evenly in the cluster.
% After the scheduling, the completion time of reduce stage is close to the optimal. \textcolor{red}{may need to add math prove between this and optimal}.
In the second round, the task --- node mapping will be adjusted according to the locality. 
The $SWAP\_TASKS$ will only be trigged when the $host\_id$ of a reduce task is not equal the $assigned\_id$.
% The closer $prob$ is to $1/m$, the more evenly this reduce partition is produced in cluster.
For a task which contains at most $prob$ data from $host$, the normalized probability $norm$ is calculated as a bound of performance degradation. We set maximum $upper\_bound$ of performance degradation equals to 10\% that can be traded for locality (in extreme skew scenarios).
Inside the $SWAP\_TASKS$, tasks will be selected and swapped without exceeding the $upper\_bound$ of each host. 
Combining these information helps the scheduler make a more balanced task --- node mapping than the na\"{i}ve Spark FIFO scheduling algorithm. 
We use the OpenCloud trace to evaluate Heuristic MinHeap. Without swapping, the Heuristic MinHeap can achieve a better performance improvement (average 5.7\%) than the default Spark FIFO scheduling algorithm (average 2.7\%).
This is the by-product optimization harvested from shuffle size prediction.

\subsubsection{Cope with Multiple Shuffles}
Multiple shuffles commonly exist in modern DAG computing. The techniques mentioned in Section \ref{shuffleprediction} can only handle the ongoing shuffle. For those pending shuffles, it is impossible to predict the sizes. This dilemma can be relieved by having all map tasks of shuffle to be scheduled by DAG framework simultaneously. But doing this introduces large overhead such as extra task serialization. To avoid violating the optimization from framework, we present Accumulate Heuristic Scheduling algorithm (Algorithm \ref{mhminheap}) to cope with multiple shuffles.

As illustrated in $M\_SCHEDULE$, the size of reduce on each node of previously scheduled $shuffles$ are counted. When a new shuffle starts, the $M\_SCHEDULE$ is called to schedule the new one accumulatively. The $size$ of each reduce and its corresponding $porb$ and $host$ in $p\_reduces$ are updated with data of $shuffles$ before $SCHEDULE$ is called. When the new task --- node mapping is available, if the new $assigned\_host$ of a reduce does not equal to the original one, the re-shuffle will be triggered to transfer data to a new host for further computing. This re-shuffle is rare since the previously shuffled data in a reduce partition contributes a huge composition. It means in the schedule phase, the $SWAP\_TASKS$ can help revise the scheduling to match the previous mapping as much as possible while maintaining the good load balance.

% As for the input of $SCHEDULE$ function in Algorithm \ref{hminheap}, $m$ is the partition number of input data; $h$ is the array of nodes ID in cluster; $p\_reduces$ is the predicted reduce matrix. Each row in $p\_reduces$ contains $rid$ as reduce partition ID; $size$ as predicted size of this partition; $prob$ as the maximum composition portion of reduce data; $host\_id$ as the node ID that produces the maximum portion of reduce data, and $assigned\_id$ as the node ID assigned by Algorithm \ref{hminheap}. As for $M$, it is consists $host\_id$, an array of $reduce$ scheduled on this node and the $size$ of them.

%# -*- coding: utf-8-unix -*-
%%==================================================
%% chapter02.tex for SJTU Master Thesis
%%==================================================

\chapter{SCache的实现}
\label{chap:impl}

本章将展现SCache系统的实现概况。
SCache是一个开源的shuffle数据管理系统,并且提供了一个对于DAG任务预调度的附属调度器。
同时SCache还在设计时提供了跨框架的接口,来实现对于现有主流分布式DAG计算框架的shuffle优化。
在这次实现中,我们以Spark作为DAG计算框架的实例来阐述在SCache辅助下DAG的新的计算流程。
我们首先在章节\ref{sec:overview}中介绍了系统设计的概要。
之后的两个章节主要介绍应用SCache优化上工程上的开销和SCache在容错性上的取舍。

\section{系统设计概要}
\label{sec:overview}

SCache在系统层面上主要包含三个部件:一个分布式的shuffle数据管理系统,一个DAG的附属调度器,和一个Spark系统内部的守护进程。
如图\ref{fig:arch}所示,SCache采用了类似于GFS\cite{gfs}经典的主从节点架构来实现对shuffle数据在集群中的管理。

\begin{figure}[!htp]
	\centering
	\includegraphics[width=0.8\textwidth]{../../PPoPP-2018/fig/arch.pdf}
	\bicaption[fig:arch]{SCache系统架构示意图}{SCache系统架构示意图}{Fig}{SCache Architecture}
\end{figure}

SCache的主节点负责通过DAG的附属调度器获取Spark上连接shuffle的reduce阶段任务预调度信息,任务的调度执行顺序等。SCache的主节点会根据reduce阶段任务的预调度信息通知到各个从属工作节点。
当工作节点收到任务预调度信息之后,会将本地已经缓存的map任务输出的shuffle数据分别发送到目的节点。
并且对于未完成的map任务,一旦工作节点通过内存拷贝的方式获取了相应的shuffle数据后就立即通过网络将shuffle数据分发出去。
除此之外,SCache主节点还会根据任务的执行顺序信息给各个shuffle的数据单元标注上优先级并且发送给各个工作节点。

SCache的从节点会在本地占用一部分内存空间用来存储shuffle的数据块。
为了减少内存管理的开销,SCache使用了Java虚拟机的堆外内存来对shuffle的数据块进行存储。
采用这种方式,既可以减少序列化反序列化带来的计算资源开销,同时又减轻了Java虚拟机垃圾回收算法(Garbage Collection)中的计算复杂度,有利于提升系统的整体性能。

于此同时,当本地的内存空间不够时,各个工作节点会根据从主节点收到的shuffle数据块优先级信息,结合本地的缓存信息(比如是否存在不完整的shuffle存储单元)来将一部分shuffle数据先保存到磁盘上。
本地的工作节点会在内存中至少缓存一个优先级最高的reduce任务需要的shuffle数据。
当本地工作节点的shuffle缓存数据被任务消耗时,该部分内存空间就会被释放,而在磁盘中缓存的较高优先级的数据就会被立刻放入内存中。
通过结合主节点的优先级信息,本地shuffle的缓存状况以及Spark任务对shuffle数据的访问状况,工作节点的调度可以保证相对独立的完成内存管理并且保证:(1)shuffle数据可以在reduce任务开始执行前就被缓存在内存当中并且(2)shuffle数据的内存缓存不会破坏全部或没有以及应用优先级的限制。

SCache中的DAG附属调度器主要负责从Spark的Driver中获取DAG的信息,包括map阶段和reduce阶段中的shuffle数目,map任务的个数和reduce任务的个数以及当前map任务的shuffle输出的数据分别或者采样任务之后的数据分布。
附属调度器会根据以上信息采用相应的线性回归算法或者采样算法来预测shuffle数据的分布,同时调用算法\ref{mhminheap}和算法\ref{hminheap}来作出一个启发式的调度。
在获得最终调度结果之后,附属调度器会将调度结果发送给SCache的主节点。
同时该调度结果也会在Spark任务调度器调度reduce阶段的任务之前将预调度的结果强制到任务调度器上。

守护进程以一个独立的线程的形式存在于Spark的内存空间,通过RPC的方式与SCache进行通信。
并且向Spark系统内部的守护进程则负责向Spark的任务和Driver提供相应的API。

\begin{table}[!hpb]
    \centering
    \bicaption[tab:apis]{SCache编程接口列表}{SCache编程接口列表}{Table}{API list of SCache}
    \begin{tabular}{ | m{2.5cm} | m{8cm} | m{5cm} | }
        \hline
        接口 & 参数 & 作用 \\ [0.5ex]
        \hline
        \hline
        registerShuffles & jobId: Int, shuffleIds: Array[Int], maps: Array[Int], reduces: Array[Int], partitioner: Array[String] & 向SCache注册shuffle \\ \hline
        getBlock & blockId: String & 向SCache获取shuffle的数据块 \\ \hline
        putBlock & blockId: String, data: Array[Byte], len: Int & 向SCache发送shuffle数据块 \\ \hline
        getShuffleStatus & jobId: Int, shuffleId: Int & 向SCache获取reduce任务的预调度结果 \\ \hline
        sampling & jobId: Int, shuffleId: Int, distribution: Array[Array[Float]] & 向SCache发送采样分布 \\ 
        \hline
    \end{tabular}
\end{table}

\section{工作流程}

接下来我们将介绍在SCache的协同下Spark执行DAG的工作流程。
工作流程的第一阶段如图\ref{fig:regshuffle}所示,SCache需要首先从Spark中获取DAG的具体信息。

\begin{figure}[!htp]
	\centering
	\includegraphics[width=0.8\textwidth]{shuffleregflow.pdf}
	\bicaption[fig:regshuffle]{注册shuffle时序图}{注册shuffle时序图}{Fig}{Sequence Diagram of Shuffle Registration}
\end{figure}

当一个Spark的工作启动时,首先会根据用户的代码生成一个关于RDD(Resilient Distributed Datasets)的系带关系(lineage)。
之后Spark的调度器会从最终的用户RDD递归向前寻找依赖的RDD。
在RDD之间的数据依赖中,如果存在部分依赖,也就是shuffle依赖,Spark会在此处插入一个shuffle过程,并且将之前的所有RDD合并成一个计算阶段(stage)。
递归寻找的过程会在当一个RDD的数据已经被计算或者已经到了存储系统的部分就会停止。
而这些计算阶段则最终组成了计算过程中的DAG逻辑。

对于DAG中相邻计算阶段之间的shuffle依赖,它们会被打包成一个RPC的调用提交到SCache的从属调度器上。如表\ref{tab:apis}中,一个shuffle依赖需要包含一个唯一的整数ID代表该工作,同时需要包含这个shuffle依赖中每个shuffle的ID,以及它们对应的分区函数的类型,map阶段任务数和reduce阶段的任务数。
在收到一次RPC提交之后,SCache的从属调度器会首先检查分区函数的类型,如果不是哈希分区函数,就会通过在Spark的Driver上的守护进程在Spark执行该计算阶段前插入一段采样程序。
我们会在章节\ref{sec:sampling}中详细阐述这个过程。

在之后的map计算阶段任务执行过程中的时序图如图\ref{fig:putblock}所示。

\begin{figure}[!htp]
	\centering
	\includegraphics[width=0.8\textwidth]{putblock.pdf}
	\bicaption[fig:putblock]{Map阶段工作时序图}{Map阶段工作时序图}{Fig}{Sequence Diagram of Map Stage}
\end{figure}

对于一个哈希分区函数的map任务,当计算结束之后分区函数会将计算产生的键值对通过相应的哈希函数划分成不同的数据块。
对于每一个数据块,SCache在Spark系统中的daemon程序都会根据其所在的工作ID对应的shuffle依赖ID,map任务ID,和reduce任务ID对每个数据块进行一个唯一的编号(即\ref{tab:apis}中的blockId)。
在此之后daemon程序会首先通过远程过程调用(Remote Procedure Call)的方式将这些shuffle数据块的元数据发送给SCache本地节点的工作进程。
于此同时,daemon进程也会通过Java对象的序列化的方式,将相应的数据序列化成字节码。
当接受到元数据之后,SCache的本地进程就会通过内存拷贝的形式,将序列化之后的字节码数据从Spark执行器的Java虚拟机的内存空间通过daemon程序拷贝到外部SCache的内存空间。
一旦内存拷贝结束,该map任务所占用的计算资源(slot)就会被立即释放。
通过SCache此处的内存拷贝优化,使得原先阻塞的磁盘I/O操作被避免,从而实现了map任务端更细粒度的硬件资源管理,提高了硬件资源的利用率和复用率。

在SCache的本地进程获取了该map任务输出的shuffle数据块之后,它会通过SCache系统内部的RPC接口通知SCache的主节点,并且在通知中附上了该map任务对于所有reduce任务产生的shuffle数据块大小(如图\ref{fig:shuffle}中的“map output”)。
每个数据块的大小可以通过内存拷贝时每个“blockId”对应的字节数组的长度来获得(即接口\ref{tab:apis}中putBlock的参数data: Array[Byte])。
当SCache的主节点收到了足够多的shuffle数据块信息之后(该threshold可以通过配置文件更改),会通过线性回归模型首先对reduce任务的输入数据分布进行预测。
之后DAG从属调度模块会根据预测数据,调用算法\ref{mhminheap}和算法\ref{hminheap}来对redcue任务进行预调度。

在SCache的中心节点完成预调度之后的流程具体如图\ref{fig:preschedule}所示。

\begin{figure}[!htp]
	\centering
	\includegraphics[width=0.8\textwidth]{preschedule.pdf}
	\bicaption[fig:preschedule]{预调度与预取工作时序图}{预调度与预取工作时序图}{Fig}{Sequence Diagram of Pre-scheduling and Pre-fetching}
\end{figure}

当预调度算法运行结束之后,SCache的主节点会将reduce任务与计算节点的映射关系进行广播。
每个工作节点的SCache进程收到预调度的信息之后,会根据节点信息做出筛选,选出需要在本地执行的reduce任务。
此后本地节点会根据相应的工作ID,shuffle依赖ID,map任务ID,以及即将在本地执行的reduce任务ID组合而成的blockId,查询获取该数据块所在的节点位置,之后便开始从该远程节点预取数据。
当一个数据块被预取之后,其在节点的内存缓存就会被释放。

在reduce任务被调度的阶段,为了使Spark任务调度器遵循SCache的预调度结果,我们修改Spark的DAG调度器,通过表\ref{tab:apis}中getShuffleStatus的接口获取SCache预调度的任务--节点映射关系。
在此基础上,我们将reduce任务中分配的节点修改成预调度的结果,同时对每个任务对节点的优先级设置为$NODE\_LOCAL$\cite{sparksource}来使得该任务被强制分配到预调度的节点,从而获取shuffle数据预取与内存缓存的优化。

\section{水塘抽样}
\label{sec:sampling}

在上文的工作流程中,如果在注册shuffle时SCache检测到不是哈希分区函数,则SCache部署在Spark主节点上的守护进程会在使用非哈希分区函数的RDD中调用其$mapPartitionsWithIndex$\cite{sparksource}方法对每个分区的数据进行水塘采样。
如图\ref{fig:sample}所示,对于每个计算节点的本地采样程序,我们使用了经典的水塘采样算法\cite{reservoir},对每个数据分区中的输入数据进行随机采样计算并且统计经过分区函数分区后的输出shuffle数据分布。
在具体实现中,对于采样的样本数,我们将其设置为$3 \times $数据块数目,以此来平衡采样的精确度和采样的开销。
此处采样的样本数可以根据用户配置进行调整。
在对任务数据进行采样的同时,算法也会统计该数据分区的大小,并以此作为该任务分区采样数据的权重。
最终这些数据会在SCache主节点进行汇总,并根据公式\ref{eq:sample}来得出每个map任务对应每个
reduce任务所产生的数据块大小,之后主节点会根据reduce的任务ID进行汇总,最终得出对于下一轮reduce任务输入数据体积分布的预测。
获取预测数据之后,SCache主节点会同上文的工作流程中一致,即调用算法\ref{mhminheap}和算法\ref{hminheap}来进行任务预调度以及后续的shuffle数据预取等工作。

\begin{figure}[!htp]
	\centering
	\includegraphics[width=0.6\textwidth]{../../PPoPP-2018/fig/sample.pdf}
	\bicaption[fig:sample]{对于一个任务数据分区的采样}{对于一个任务数据分区的采样}{Fig}{Reservoir Sampling of One Data Partition}
\end{figure}

\section{容错机制}

本小结将从两个角度来讨论SCache在容错性上的设计:Spark执行器或者SCache进程的崩溃造成的错误以及节点崩溃造成的节点下线的错误。

对于Spark执行器发生的错误,SCache采用了独立于Spark执行器的内存空间进行shuffle数据的存储和管理,因此即使Spark的执行器因为错误或者调度原因被Spark主节点杀死或者重启,其中所执行的任务需要的shuffle数据仍然不会遗失。

对于SCache进程的容错性,SCache采用了本地磁盘备份的机制来防止因为错误造成的shuffle数据损失。
当shuffle数据块预取的同时,无论该数据块是否在内存中有缓存,SCache都会在节点本地通过异步I/O的方式在本地磁盘对该数据块进行备份。
这些备份会在shuffle数据块被任务使用之后释放。
同时,SCache的工作节点会定期向主节点发送心跳信息,一旦主节点没有收到其中一个工作节点的心跳,就可以通过远程脚本控制其在本地重启。
当工作节点重启之后,主节点会向其发送当前shuffle的调度状态,工作节点则会根据本地磁盘的shuffle数据块备份状况和当前的shuffle状态来判断是否有未完成的shuffle数据块预取以及缓存任务。
而在工作节点进程失效的过程中,Spark的执行器如果有相应的reduce任务执行需要数据,则会被其中的守护进程阻塞住,直到本地工作进程完成重启。
对于SCache主节点的容错性,由于主节点的进程只服务与一个Spark的driver程序,并且不需要调度对于跨工作的shuffle数据依赖,因此主节点采用了简单的本地磁盘备份日志的方式来备份当前对于相关shuffle的元数据记录和调度记录等信息。

为了降低系统复杂度,SCache在设计的过程中并未引入应对节点崩溃下线的容错机制。
省略这部分机制的原因主要有一下几点:
\begin{itemize}
    \item 备份机制的开销过大,与优化shuffle过程的性能目标相对立。在设计系统的过程中我们并未加入比如经典的GFS\cite{gfs}中提出的三备份机制。
    采用此类容错机制虽然能提升系统的鲁棒性,但是同时也会增加系统在执行shuffle过程中的开销。
    比如如果采用三备份机制,那么在shuffle数据预取阶段以及本地shuffle存储的开销都会随着备份数目的增加而倍增。
    虽然在章节\ref{subsec:size}中我们发现shuffle数据体积对较小,但是备份机制带来的额外开销无疑会给数据中心网络的性能甚至内存和硬盘的性能带来一定挑战,这也与我们的优化目标相违背。
    \item 需要备份的数据没有复用价值。对于每一个shuffle依赖中的数据,只会在DAG计算过程中被使用一次,之后便会被释放。因此对于这些数据的备份就不像一些复用性较强的存储系统甚至Spark本身的RDD来的重要。
    \item SCache与计算框架的共生性。为了提高效率,SCache采用了与DAG计算框架一一对应的共生设计,即每个计算节点既有DAG计算框架的执行器,又有SCache的工作进程。
    这种共生的模式也意味着当该节点失效时,SCache的工作进程和DAG的计算执行器会同时失效。而DAG计算框架本身又有不同的容错机制来保证任务执行。
    而此时对节点进行备份机制的设计可能不但不利于计算的快速恢复,反而会因为不同的容错策略导致SCache针对节点的荣作机制变成无效操作。
    比如在Spark中,会采取对失效的RDD分区进行并行恢复的模式,在此过程中原来属于一个数据分区的数据会进行再分区,从而加快恢复速度。
    那么此时针对SCache单节点的容错机制即使恢复了丢失的shuffle数据,该份数据在计算中也没有使用价值。
\end{itemize}

虽然在设计中省略了对节点下线的容错性,但是为了保证DAG计算过程在发生错误时仍然能够正确执行,SCache在此处采用了借助DAG计算框架容错性的恢复模式。
比如在上文提到的Spark恢复模式中,该失效的RDD分区会进行一个重新分区的并行计算,而在次过程中,SCache对于shuffle的优化将对这部分逻辑进行重新提交和分配。
通过借助DAG计算框架本身的容错性,SCache能保证在节点下线的恢复过程中不破坏任务的正确性,同时提供对恢复中的shuffle优化。

\section{普适性分析}

为了验证本研究方案的普适性,本章将详细展示在工程方面改造Spark的开销。
同时也将展示在Hadoop上的可行性分析。
通过以上两个方面的分析和探索,最终可以证明SCache提供的优化方案可以在不需要大量改造成本的情况下被目前主流的分布式DAG计算框架应用。

\subsection{Spark的实现分析}
本次实现中,我们利用Spark这个被广泛应用的分布式DAG计算平台来对SCache的优化进行适配。
在Spark上的修改主要集中在以下几个方面。

首先,我们通过netty\cite{netty}的库在Spark内部实现了一个守护进程--$ScacheDaemon$。
在$ScacheDaemon$中,我们实现了表格\ref{tab:apis}中的相应接口,具体代码片段如代码\ref{code:daemon}所示。
其中变量$daemon$是SCache在本地的工作进程,可以通过读取配置文件来获取具体端口,来建立连接,$daemon$本身是一个netty的RPC参考(Reference)实例。
实现中的所有接口都会通过$daemon$这个RPC的参考实例与本地的工作进程和SCache主节点进行数据交互。

\begin{lstlisting}[style={myScalastyle}, caption={ScacheDaemon代码片段}, label={code:daemon}]
    private[spark] class ScacheDaemon (conf: SparkConf) extends Logging {

        val scacheHome = conf.get("spark.scache.home", "~/SCache")
        val platform = "spark"  
        val daemon = new Daemon(scacheHome, platform)   
        val reduceStatus = new ConcurrentHashMap[(Int, Int), mutable.HashMap[Int, Array[String]]]() 
        private var runningJId: Int = -1    
        def setRunningJId(jid: Int): Unit = {
          //Set current runing job id
        }   
        def getRunningJId: Int = runningJId 
        def putBlock (blockId: BlockId, data: Array[Byte], rawLen: Int, compressedLen: Int): Boolean = {
          // Called by Spark Executor
          // The block data is transfered from JVM space of Spark Executor
        }   
        def getBlock(blockId: BlockId): Option[Array[Byte]] = {
          // Called by Spark Executor
          // The block data is fetched from memory of SCache to Spark Executor
        }   
        def registerShuffles(jobId: Int, shuffleIds: Array[Int], maps: Array[Int], reduces: Array[Int], partitioner: Array[String]): Unit = {
          // Called by Spark DAG Scheduler
          // Register shuffle to SCache
        }   
        def getShuffleStatus(jobId: Int, shuffleId: Int): mutable.HashMap[Int, Array[String]] = {
          // Called by Spark Task Scheduler
          // Get pre-scheduled reduce tasks
        }   
    }
\end{lstlisting}

其次,我们修改了Spark的$DAGScheduler$来实现对DAG信息的获取。
具体可以参考如代码\ref{code:dagScheduler}。
在Spark的Driver节点上也存在相应的$Daemon$进程。
当Spark的$DAGScheduler$在从最终的RDD递归向前建立DAG的过程中($getParentStages$函数),如果发现RDD之前的依赖关系是shuffle依赖,则通过$env.scacheDaemon.registerShuffles$
获取本地已经初始话好的守护进程实例,并通过其向SCache提交该shuffle依赖的元数据。

\begin{lstlisting}[style={myScalastyle}, caption={DAGScheduler代码片段}, label={code:dagScheduler}]
    private[spark] class DAGScheduler(...) extends Logging {
            // Skip
            /**
            * Get or create the list of parent stages for a given RDD.  The new Stages will be created with
            * the provided firstJobId.
            */
           private def getParentStages(rdd: RDD[_], firstJobId: Int): List[Stage] = {
             val parents = new HashSet[Stage]
             val visited = new HashSet[RDD[_]]
             val waitingForVisit = new Stack[RDD[_]]
             def visit(r: RDD[_]) {
               if (!visited(r)) {
                 visited += r
                 val shuffles = new ArrayBuffer[ShuffleDependency[_, _, _]]
                 for (dep <- r.dependencies) {
                   dep match {
                     case shufDep: ShuffleDependency[_, _, _] =>
                       parents += getShuffleMapStage(shufDep, firstJobId)
                       shuffles.append(shufDep)
                     case _ =>
                       waitingForVisit.push(dep.rdd)
                   }
                 }
                 if (!shuffles.isEmpty && sc.getConf.getBoolean("spark.scache.enable", false)) {
                   env.scacheDaemon.registerShuffles(firstJobId, shuffles.toArray, rdd.partitions.length, rdd.partitioner.get.toString)
                 }
               }
             }
             waitingForVisit.push(rdd)
             while (waitingForVisit.nonEmpty) {
               visit(waitingForVisit.pop())
             }
             parents.toList
           }
    }
\end{lstlisting}

同时,为了使Spark的执行器在执行map任务与reduce任务时能够和SCache进行数据交互,我们重新实现了一个$ScacheBlockObjectWriter$用来实现map任务的shuffle数据向SCache写入,
实现了$ScacheBlockTransferService$实现了reduce任务向SCache的shuffle数据读取。
$ScacheBlockObjectWriter$实现了对Spark原来的$DiskBlockObjectWriter$的继承,并且重载了其中的方法。
代码\ref{code:writer}展示了$ScacheBlockObjectWriter$在继承过程中比较重要的几个方法重载。
在这里去掉了原先的磁盘写操作,取代的是对内存缓存空间的数据写入以及内存拷贝,从而优化了map计算任务的磁盘阻塞。

\begin{lstlisting}[style={myScalastyle}, caption={ScacheBlockObjectWriter代码片段}, label={code:writer}]
    class ScacheBlockObjectWriter (...)extends DiskBlockObjectWriter(...) with Logging{
      override def open(): DiskBlockObjectWriter = {
        // Open a byte stream buffer
      }
      override def close(): Unit = {
        // Do memory copy and free memory
      }
      override def write(key: Any, value: Any): Unit = {
        // Write data to byte stream buffer
      }
    }
\end{lstlisting}

而在reduce阶段,我们修改了$ShuffleBlockFetcherIterator$中获取shuffle数据的方法,即调用$ScacheBlockTransferService$中的相应接口来从SCache获取shuffle的数据块。
虽然此处的数据访问通过内存拷贝来完成,但是为了进一步提高性能,以及预防在网络极端拥塞的情况下,最后一轮的几个map任务的输出shuffle数据仍然没有传输完成,在向SCache获取shuffle数据块的时候,
采用了多线程异步非阻塞的方式。
这种实现方式在第一时间能将已经缓存的shuffle数据块返回给Spark的reduce任务进行计算,因而能将获取shuffle数据时产生对计算的阻塞的可能性降到最低。
代码片段如代码\ref{code:reader}所示。

\begin{lstlisting}[style={myScalastyle}, caption={ScacheBlockTransferService代码片段}, label={code:reader}]
  class ScacheBlockTransferService(daemon: ScacheDaemon) extends Logging{
    def fetchBlocks(
        host: String,
        port: Int,
        execId: String,
        blockIds: Array[String],
        listener: BlockFetchingListener): Unit = {
      // Fetch block from SCache
      // It is a multi-thread asynchronous method that can fetch multiple blocks simultaneously
    }
  }
\end{lstlisting}

在reduce任务预调度方面,Spark的\verb|DAGScheduler.scala|类使其在生成reduce任务时先通过调用RDD的\verb|getPreferedLocs|方法来查询改RDD每个分区对节点的偏好。
而我们通过修改\verb|ShuffleRDD.scala|的\verb|getPreferedLocs|方法,将SCache预调度的结果通过接口\ref{tab:apis}中\verb|getShuffleStatus|返回给RDD。
具体过程可以参考代码\ref{code:preschedule}
获取预调度结果之后,Spark的\verb|TaskSchedulerImpl.scala|就会在调度reduce任务时将其分配到已经完成shuffle数据预取的节点上,从而获得SCache的优化。

\begin{lstlisting}[style={myScalastyle}, caption={Reduce预调度代码片段}, label={code:preschedule}]
    class ShuffledRDD[...](...){
        //Skip
        override protected def getPreferredLocations(partition: Partition): Seq[String] = {
            if (SparkEnv.get.conf.getBoolean("spark.scache.enable", false)) {
              val dep = dependencies.head.asInstanceOf[ShuffleDependency[K, V, C]]
              // Ask SCache to get preferred location
              val locs = SparkEnv.get.scacheDaemon.getShuffleStatusForPartition(dep.shuffleId, partition.index)
              locs.toSeq
            } else {
              val tracker = SparkEnv.get.mapOutputTracker.asInstanceOf[MapOutputTrackerMaster]
              val dep = dependencies.head.asInstanceOf[ShuffleDependency[K, V, C]]
              tracker.getPreferredLocationsForShuffle(dep, partition.index)
            }
        }
    }
\end{lstlisting}

在插入采样程序部分,我们利用了Spark中$RangePartitioner$在决定分区边界时对RDD进行计算的过程,在其中插入了采样过程。
为了提高效率,对RDD数据的采样以及分区的计算是在Spark的Driver内部进行。
计算完成之后再将数据发送给SCache,具体逻辑如代码\ref{code:sample}所示。
其中返回的$distribution$将会作为采样的分布通过接口\ref{tab:apis}中的$sampling$发送个SCache。

\begin{lstlisting}[style={myScalastyle}, caption={水塘采样代码片段}, label={code:sample}]
    def determineBounds[K : Ordering : ClassTag](
        candidates: ArrayBuffer[(K, Float, Int)],
        partitions: Int,
        depPartitionNum: Int): (Array[K], Array[Array[Int]]) = {
      // Skip
      val ordered = candidates.sortBy(_._1)
      val distribution = Array.fill[Int](depPartitionNum)(0)
        .map(x => new Array[Float](partitions))
      var i = 0
      var j = 0
      var previousBound = Option.empty[K]
      while ((i < numCandidates) && (j < partitions - 1)) {
        val (key, weight, index) = ordered(i)
        cumWeight += weight
        distribution(index)(j) += weight
        if (cumWeight >= target) {
          // Skip duplicate values.
          if (previousBound.isEmpty || ordering.gt(key, previousBound.get)) {
            bounds += key
            target += step
            j += 1
            previousBound = Some(key)
          }
        }
        i += 1
      }
      while (i < numCandidates) {
        // calculate the distribution of last partition
        val (key, weight, index) = ordered(i)
        distribution(index)(j) += weight
        i += 1
      }
      (bounds.toArray, distribution)
    }
\end{lstlisting}

以上四个部分是在Spark适配过程中做出的主要改动,其代码量大约在1000行左右,相较于Spark本身几十万行的代码量,可以说这个改动是非常的小。

\subsection{Hadoop MapReduce的可行性分析}

受限于时间,本研究并没有对Hadoop MapReduce进行SCache的适配改造。
但是为了验证SCache对shuffle的优化的普适性,本研究仍对Hadoop MapReduce的适配可行性进行了分析研究。

首先,我们对Hadoop MapReduce的计算过程中存在的shuffle特点进行分析。
虽然在表达DAG的过程中Hadoop MapReduce与Spark存在较大差异,但是在每个map和reduce的计算阶段之间都会存在一个shuffle的过程却十分接近。
在执行map任务时,Hadoop MapReduce中的\verb|MapTask.java|类
会将map任务计算产生的shuffle结果通过\verb|MapOutputClollector.java|接口写入到本地磁盘进行保存。
当reduce任务数目不为0时,即该map阶段不是工作的最后一次计算过程时,map任务会调用\verb|NewOutputCollector.java|来写入该阶段的输出键值对。
在具体执行过程中,其通过实例化一个\verb|collector|来进行写操作,而该\verb|collector|就是一个\verb|MapOutputClollector.java|的具体实现。
具体代码参考代码片段\ref{code:hadoopmap}

在reduce阶段,每个reduce任务由类\verb|ReduceTask.java|来执行。
具体来说,每个reduce任务会通过\verb|ShuffleConsumerPlugin|的接口来实现第一阶段的远程shuffle数据拷贝过程。
在目前的MapReduce中,该接口由类\verb|Shuffle<K, V>|实现了一个多线程的异步远程shuffle数据读取。
具体可以参考代码片段\ref{code:hadoopreduce}。

\begin{lstlisting}[language={Java}, caption={Hadoop MapReduce中Map阶段的shuffle写代码片段}, label={code:hadoopmap}]
    private <...>
    void runNewMapper(...) throws IOException, ClassNotFoundException,
                             InterruptedException {
        // Skip
        // get an output object
        if (job.getNumReduceTasks() == 0) {
          output = 
            new NewDirectOutputCollector(taskContext, job, umbilical, reporter);
        } else {
          output = new NewOutputCollector(taskContext, job, umbilical, reporter);
        }
        // Skip    
        try {
          long startTime = System.currentTimeMillis();
          input.initialize(split, mapperContext);
          mapper.run(mapperContext);
          mapPhase.complete();
          setPhase(TaskStatus.Phase.SORT);
          statusUpdate(umbilical);
          input.close();
          input = null;
          output.close(mapperContext);
          output = null;
        } finally {
          closeQuietly(input);
          closeQuietly(output, mapperContext);
        }
    }

    NewOutputCollector(...) throws IOException, ClassNotFoundException {
        @Override
        public void write(K key, V value) throws IOException, InterruptedException {
          collector.collect(key, value,
                            partitioner.getPartition(key, value, partitions));
        }
        @Override
        public void close(TaskAttemptContext context
                          ) throws IOException,InterruptedException {
          try {
            collector.flush();
          } catch (ClassNotFoundException cnf) {
            throw new IOException("can't find class ", cnf);
          }
          collector.close();
        }
    }
\end{lstlisting}

\begin{lstlisting}[language={Java}, caption={Hadoop MapReduce中Reduce阶段的shuffle读代码片段}, label={code:hadoopreduce}]
    public class ReduceTask extends Task {
        public void run(...) throws IOException, InterruptedException, ClassNotFoundException {
          // Skip
          ShuffleConsumerPlugin.Context shuffleContext = new ShuffleConsumerPlugin.Context(...);
          shuffleConsumerPlugin.init(shuffleContext);
          // fetch the remote shuffle blocks simultaneously
          rIter = shuffleConsumerPlugin.run();
        }
    }

    public class Shuffle<K, V> implements ShuffleConsumerPlugin<K, V>, ExceptionReporter {
        // Skip
        @Override
        public RawKeyValueIterator run() throws IOException, InterruptedException {
          //Skip
          if (isLocal) {
            fetchers[0] = new LocalFetcher<K, V>(jobConf, reduceId, scheduler,
            merger, reporter, metrics, this, reduceTask.getShuffleSecret(),
            localMapFiles);
            fetchers[0].start();
          } else {
            for (int i=0; i < numFetchers; ++i) {
              fetchers[i] = new Fetcher<K,V>(...);
              fetchers[i].start();
            }
          }
        }
    }
\end{lstlisting}

而对于任务的调度,在Hadoop MapReduce中,主要由一系列状态机的转移和Yarn的资源调度机制协同完成。
具体来讲,在一个MapReduce任务被提交之后,经过一系列的状态转移,最终会由\verb|JobImpl|类来完成map和reduce阶段的具体任务生成以及提交。
而由于MapReduce的计算DAG相较于Spark十分简单,即在一个任务只有Map和Reduce两个过程,所以对于shuffle的相关元数据便可以在此处获得。
具体关于shuffle两端map和reduce任务的生成,可以参考代码\ref{code:hadoopshuffle}。

\begin{lstlisting}[language={Java}, caption={Hadoop MapReduce中shuffle元数据生成代码片段}, label={code:hadoopshuffle}]
    public class JobImpl implements org.apache.hadoop.mapreduce.v2.app.job.Job, EventHandler<JobEvent> {
        // Skip
        @Override
        public JobStateInternal transition(JobImpl job, JobEvent event) {
          // Skip
          createMapTasks(job, inputLength, taskSplitMetaInfo);
          createReduceTasks(job);
        }
    }
\end{lstlisting}

对于reduce阶段的任务预调度,在MapReduce中都是通过\verb|RMContainerAllocator.java|的类来实现的。
当其收到对reduce任务的调度的请求时,\verb|RMContainerAllocator.java|会根据从Yarn的ResourceManager那里获取的可用容器数(Container)进行分配。
具体可以参考代码片段\ref{code:hadoopschedule}

\begin{lstlisting}[language={Java}, caption={Hadoop MapReduce中reduce任务调度代码片段}, label={code:hadoopschedule}]
    public class RMContainerAllocator extends RMContainerRequestor implements ContainerAllocator {
        private ContainerRequest assignToReduce(Container allocated) {
          ContainerRequest assigned = null;
          //try to assign to reduces if present
          if (assigned == null && reduces.size() > 0 && canAssignReduces()) {
            TaskAttemptId tId = reduces.keySet().iterator().next();
            assigned = reduces.remove(tId);
            LOG.info("Assigned to reduce");
          }
          return assigned;
        }
    }
\end{lstlisting}

综上所述,我们给出了一下对Hadoop MapReduce的SCache适配的改造方案。
\begin{enumerate}
    \item Shuffle元数据获取:在\verb|JobImpl|类中发生任务提交而产生的状态转换函数执行过程中,可以获取shuffle两端对应的map任务个数,reduce任务个数等信息。
    此时,便可以将这些数据通过接口\ref{tab:apis}中的\verb|registerShuffles|进行提交。
    \item Map任务shuffle数据的解耦:可以通过重新实现一个对接口\verb|MapOutputClollector|的实现类来获取shuffle数据,并且通过内存拷贝传递给SCache。
    \item Reduce任务shuffle数据的解耦:可以通过重新实现一个接口\verb|ShuffleConsumerPlugin|的实现类来从SCache的内存缓存获取shuffle的数据。
    \item Reduce任务的预调度:需要将SCache的预调度结果通过接口在\verb|RMContainerAllocator.java|调度任务前对预调度结果进行获取,然后在其调用\verb|assignToReduce|函数时遵循预调度的结果即可。
    \item 采样任务的实现:在\verb|JobImpl|类中生成具体任务的时候,可以通过配置信息来获取分区函数的信息。
    如果发现是一个非哈希的分区函数,则在创建map任务的同时创建每个任务对应的采样任务对其进行数据采样。
\end{enumerate}

以上两个章节通过对目前最流行的Spark和Hadoop MapReduce这两个典型的分布式DAG计算框架应用SCache优化的可行性做出了具体的分析。
通过展现Spark上的具体实现技术,可以看到应用SCache的优化对于原有框架本身并不需要做出巨大的修改。
同时,通过对Hadoop MapReduce的任务执行框架任务调度流程,shuffle过程的分析,我们发现对于SCache提供的优化策略在Hadoop MapReduce上也能非常方便的适配。
因此我们认为SCache提供的优化方案对于目前主流的分布式DAG计算框架具有较好的普适性。
能够通过对现有框架较少的改动而实现适配与shuffle的优化。
%# -*- coding: utf-8-unix -*-
%%==================================================
%% chapter02.tex for SJTU Master Thesis
%%==================================================

\chapter{实验分析}
\label{chap:evaluation}
% \include{tex/example}
% \include{tex/faq}
%# -*- coding: utf-8-unix -*-
%%==================================================
%% conclusion.tex for SJTUThesis
%% Encoding: UTF-8
%%==================================================

\chapter{总结与展望}
\label{chap:summary}

   



\appendix	% 使用英文字母对附录编号,重新定义附录中的公式、图图表编号样式
% \renewcommand\theequation{\Alph{chapter}--\arabic{equation}}	
% \renewcommand\thefigure{\Alph{chapter}--\arabic{figure}}
% \renewcommand\thetable{\Alph{chapter}--\arabic{table}}
% \renewcommand\thealgorithm{\Alph{chapter}--\arabic{algorithm}}

%% 附录内容,本科学位论文可以用翻译的文献替代。
% \include{tex/app_setup}
% \include{tex/app_eq}
% \include{tex/app_cjk}
% \include{tex/app_log}

\backmatter	% 文后无编号部分

%% 参考资料
\printbibliography[heading=bibintoc]

%% 致谢、发表论文、申请专利、参与项目、简历
%% 用于盲审的论文需隐去致谢、发表论文、申请专利、参与的项目
\makeatletter

%%
% "研究生学位论文送盲审印刷格式的统一要求"
% http://www.gs.sjtu.edu.cn/inform/3/2015/20151120_123928_738.htm

% 盲审删去删去致谢页
\ifsjtu@review\relax\else
  %# -*- coding: utf-8-unix -*-
\begin{thanks}

在本论文完成之际,谨向我的导师戚正伟教授致以真诚的谢意。
在我的科研上,戚老师结合了我的兴趣在选题和深入研究上给了我巨大的支持和细致的指导。同时在研究生期间还鼓励和资助我去香港科技大学进行学术交流,让我有机会在科研上走得更远。
同时还要感谢香港科技大学系统与网络组的陈凯教授。在香港科技大学访问期间,陈凯老师给了我很多研究上的指导和建议。
此外还要感谢香港科技大学的白巍,张弘,陈晨,张骏雪等同学在我访问期间和后续的科研道路上给予的帮助。
当然也十分感谢我的家人以及女朋友姚岚在学习和生活上对我的关怀和帮助。
你们的支持使我在研究生学习期间充满动力和信心。
最后,还要感谢在研究生期间一起学习生活的同学和朋友们。
在此还要衷心感谢在百忙中评阅论文和参与答辩的各位专家和教授!

\end{thanks}
 	  %% 致谢
\fi

\ifsjtu@bachelor
  % 学士学位论文要求在最后有一个英文大摘要,单独编页码
  \pagestyle{biglast}
  \include{tex/end_english_abstract}
\else
  % 盲审论文中,发表学术论文及参与科研情况等仅以第几作者注明即可,不要出现作者或他人姓名
  \ifsjtu@review\relax
    %# -*- coding: utf-8-unix -*-

\begin{publications}{99}
    \item\textsc{第一作者}. {Seagull–A Real-Time Coflow Scheduling System}[C]// Cyber Security
    and Cloud Computing (CSCloud), 2015 IEEE 2nd International Conference on. IEEE. [S.l.]: [s.n.],
    2015: 540–545.
    \item\textsc{第一作者}. {Efficient Shuffle Management with SCache for DAG Computing Frameworks}[C/OL]// Proceedings of the 23th ACM SIGPLAN Symposium on Principles and Practice of
    Parallel Programming. PPoPP 2018. (已录用)
\end{publications}

    % \include{tex/projectsreview}
  \else
    %# -*- coding: utf-8-unix -*-
%%==================================================
%% pub.tex for SJTUThesis
%% Encoding: UTF-8
%%==================================================

\begin{publications}{99}
    \item\textsc{FU Z, SONG T, WANG S}, et al. {Seagull–A Real-Time Coflow Scheduling System}[C]// Cyber Security
    and Cloud Computing (CSCloud), 2015 IEEE 2nd International Conference on. IEEE. [S.l.]: [s.n.],
    2015: 540–545.
    \item\textsc{FU Z, SONG T, QI Z}, et al. {Efficient Shuffle Management with SCache for DAG Computing Frameworks}[C/OL]// Proceedings of the 23th ACM SIGPLAN Symposium on Principles and Practice of
    Parallel Programming. PPoPP 2018.
\end{publications}
	      %% 发表论文
    % \include{tex/projects}  %% 参与的项目
  \fi
\fi

%# -*- coding: utf-8-unix -*-
\begin{patents}{99}
    \item 第一发明人,一种分布式动态路由网络的错误定位方法,专利申请号201510028628.7
    \item 第一发明人,一种用于分布式计算机平台的网络流组调度方法,专利申请号201510526867.5 / PCT/CN2016/086543
    \item 第二发明人,一种云数据中心的负载预测方法,专利申请号201510658479.2
    \item 第一发明人,一种基于映射-归约计算模型的洗牌数据缓存方法,专利申请号201610712705.5
    \item 第一发明人,一种流水化数据洗牌传输的Spark任务调度与执行方法,专利申请号201610029211.7
\end{patents}
	  %% 申请专利
% \include{tex/resume}	  %% 个人简历

\makeatother

\end{document}
