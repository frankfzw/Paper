\begin{abstract}

Shuffle is the term used to describe the cross-network read and the aggregation of partitioned data from ancestral task before invoking reduce operations.
As DAG computing frameworks keep evolving, the calculation and scheduling process of each task has already been well optimized.
However,shuffle cuts off the data processing pipeline, introduce significant latency to successors.
(The shuffule process,however,still introduces significant latency to successors when it cuts off the data processing pipeline)
To remove shuffle overhead, we present SCache, a plugin system to decouple shuffle from DAG computing framework. SCache captures shuffle data in memory and employs heuristic-MinHeap scheduling to balance data blocks to eliminate the explicit barrier. We implemented SCache and changed Spark to use SCache as an external shuffle service and co-scheduler. We evaluated SCache's performance on both simulation and 50-machines Amazon EC2 cluster.
Results show that, by incorporating SCache, the shuffle overhead of Spark can be reduced in overall 90\%.

\end{abstract}
