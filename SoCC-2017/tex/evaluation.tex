\section{Evaluation}\label{evaluation}

In this section, we present the evaluation result of Spark with SCache comparing with the original Spark. We first run simple DAG computing jobs with two stages to analyze the impact of shuffle pre-fetch in the Spark cluster. The shuffle dependency between two stages contains one shuffle and two shuffles respectively. In order to prove the performance gain of SCache with a real production workload, we run a benchmark named Spark Terasort\cite{spark-tera} and TPC-DS\cite{tpcds} and evaluate the improvement of SCache for each stage and the overall performance. At last, we prsent the overhead of sampling. Because a complex Spark application consists of multiple stages. The completion time of each stage varies under different input data and configurations, as well as different number of stages. This uncertainty leads to the dilemma that dramatic fluctuation in overall performance comparing. To present a straightforward illustration, we limit the scope of most evalutions in single stages. 

\subsection{Setup}\label{stepup}
We run our experiments on a 50 m4.xlarge nodes cluster on Amazon EC2\cite{aws}. Each node has 16GB memory and 4 CPUs. The network bandwidth is not specifically provided by Amazon. Our evaluation reveals the bandwidth is about 300 Mbps(see Figure \ref{fig:util}).

\subsection{Simple DAG Analysis}

\subsubsection{Differential Runtime Hardware Utilization} 
We first run the same single shuffle test (GroupByTest from Spark example\cite{sparksource}) as we mentioned in Figure \ref{fig:util}. As shown in \textcolor{red}{Figure}, we can observe a huge overlap between CPU, disk and network. By running Spark with SCache, the overall utilization of the cluster stays in a high level. The decouple of shuffle write from map tasks free the CPU earlier, which leads to a faster task computation. The shuffle pre-fetch starts the shuffle data transfer in the early stage of map phase shift the network transfer completion time, so that the computaion of reduce can start immediately afther scheduled. And this is the main performance gain we achieved on the scope of hardware utilization by SCache.

\subsubsection{Performace Gain in Both Stages}
\subsection{Performance Gain in Production Workload}

\subsection{Overhead of Sampling}

